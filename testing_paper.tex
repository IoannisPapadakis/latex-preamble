%--------------------------------------------------------------
\documentclass[12pt]{article}

\usepackage[normalpaper]{mcpreamble}
\title{Something about fiscal relief and ARRA right?}
\author{Mauricio C\'aceres}
%\date{}

%--------------------------------------------------------------
\begin{document}
\displayoptions

\begin{abstract}
  \noindent Chodorow et al. (2012) estimate the causal effect on employment of the 2009 American Recovery and Reinvestment Act (ARRA), in particular of the $\$88$ billion in aid that went to state governments via Medicaid reimbursement. They report a cost per job-year of about $\$26,000$, which is very low when compared to estimates found elsewhere in the literature. However, their result is being driven by only one state, Washington D.C. If we exclude it from the data, the estimate is halved to $1.9$ job-years, with $1.6$ job-years created outside of government, and is no longer significant.
\end{abstract}

%--------------------------------------------------------------
\section{Introduction}
\label{sec:introduction}

The paper is estimating one main equation.
\begin{equation}
  \dfrac{E_1^s - E_0^s}{N^s}
  = \beta_0 + \boldsymbol\beta_1 \dfrac{Aid^s}{N^s} + \beta_2 ~ Controls^s + \varepsilon^s
\end{equation}
\[
  x^2 + \dfrac{\alpha + \beta + \zeta \left(\int_{\infty}^{\lim X_n \to \empty}\right)}{\bm \omega + \Omega_k \left[\mathbb{R} \supset \mathbb{N}\right]}
  \Gamma ? \Lambda\Upsilon = \varphi : \varepsilon + \rho ! \varrho
\acute ~ \grave ~ \breve ~ \check ~ \mathring ~ \mathdollar
\overline{\beta_k} \widetilde{\varphi_z} \dot{y}_t \ddot{xx}_t \widehat{\zeta}
\]

$E_i^s$ is the seasonally-adjusted employment, $N^s$ is the $16+$ population, $Aid^s$ is FMAP (Federal Medical Assistance Percentages) transfers per person $16+$ from December 2008, when it was first clear the funds would be available, through June 2010, the end of the Fiscal year in 2010 for most states.

Medicaid is used because the program operates on a mandatory basis, and increasing the Federal share of reimbursements effectively transfered money into the states' budgets for any purpose. A priori, however, transfers can have a small or negligible effect on economic outcomes if states just put the money in their rainy day funds. But if states use them to avoid tax cuts, budget cuts, or program cuts then it would stimulate the economy. ARRA increased the percentage of Medicaid expenditures government paid by 6.2\% for all states and prevented states from restricting requirements. States with larger increases in unemployment received more aid.

The authors are concerned with the fact worse off states received more funds---a simple OLS regression would understate the result. Hence they instrument FMAP transfers with Medicaid spending for the 2007 fiscal year.\footnote{We skip the authors' justification of their instrument and move to discuss their main findings.} Their baseline effect is computed for the employment change through June 2009, and this is the equation for which most of their robustness checks are implemented. They are also concerned that states that got more or less Medicaid spending might have already been in an upward or downward employment path, and add controls to account for unobservables.

Now the identifying assumption is that states with higher pre-recession Medicaid spending would not have experienced different employment outcomes. The authors include dummies for the nine census divisions and lagged employment change (the seven months prior to the recession) to control for this possibility. They note that while they cannot find evidence of such relation, adding the control for lagged employment changes the point estimate in their IV specification from $4.61$ to $2.83$, more than a standard deviation. This large drop seems to be of concern for the authors, so the specification with lagged employment is the one they prefer.

Lastly, to compute the impact of the program their approach is then to estimate that equation for each date range from December 2008 to January 2009 through December 2008 to June 2010. They estimate the effect of aid per worker to on change over the time period in seasonally adjusted employment per worker. Then the sum of the coefficients divided by $12$ will be an estimate of the permanent (for the period) effect of aid on employment. Following our discussion, their results can be summarized in two figures, numbered $3$ and $4$ in their paper.
\begin{figure}[ht!]
  \centering
  \caption{FMAP transfers IV coefficients, total (left) and government (right) employment}
  \vspace{3pt}
  %\includegraphics[width=\linewidth]{figure34_authors}
  \parbox{0.9\textwidth}{\footnotesize The outcome is change in seasonally adjusted employment between December 2008 and the month on the $x$-axis. All controls are included. Dashed lines indicate a 95\% CI from robust standard errors.}
  \label{fig:fig34}
\end{figure}


%--------------------------------------------------------------
\section{Conclusion}
\label{sec:conclusion}

Amazing Results

%--------------------------------------------------------------
\end{document}
